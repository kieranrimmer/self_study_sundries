\documentclass[12pt, letterpaper, twoside]{article}
\usepackage[utf8]{inputenc}
\usepackage{amsmath}

\title{Generating Functions and Linear Recurrences}
\author{Kieran Rimmer \thanks{just some Jimbo}}
\date{September 2020}

\begin{document}

\maketitle

Merge sort is a divide and conquer algortithm wherein a data structure to be sorted is 
recursively split in two until sorting is trivial, and then then sorted substructres are
merged until the full structure is sorted.

For simplicity, let us consider only data which is sized in powers of two.
Then the worst case running time \[ T_n \] for a naive implementation of merge 
sort can be approximated with:

\begin{equation} 
\begin{split}
T_n & = c_1 + T_{(2^n/2)} + T_{(2^n/2)} + c_2 \cdot 2^n \\
    & = c_1 + 2 \cdot T_{(2^n/2)} + c_2 \cdot 2^n \label{eq1}
\end{split}
\end{equation}


This relation yeids a sequence, which has a corresponding \textit{generating function}:

\begin{equation} \label{eq2}
<T_0, T_1, T_2, T_3 \dots > \leftrightarrow F(x) = T_0 + T_1 \cdot x + T_2 \cdot x^2 + T_3 \cdot x^3 \dots
\end{equation}


The sequence can be broken down into a sum of sequences, for which convenient
closed form summations exist \textit{if convergence is assumed}:



\begin{equation} \label{eq3}
    \begin{split}
    <c_1, c_1, c_1, c_1 \dots > & \leftrightarrow c_1 / (1-x) \\
    <c_2, 2 \cdot c_2, 4 \cdot c_2 \dots 2^n \cdot c_2 > & \leftrightarrow c_2 / (1-2x) \\
    + <0, 2 \cdot T_0, 2 \cdot T_1 \dots 2 \cdot T_{n-1} > & \leftrightarrow 2xF(x) \\
    \hline \\
    <T_0, T_1, T_2, T_3 \dots > & \leftrightarrow F(x) = c_1 / (1-x) + c_2 / (1-2x) + 2xF(x) \\
\end{split}
\end{equation}




\end{document}